\documentclass[epsfig,12pt]{article}
\usepackage{epsfig}
\usepackage{graphicx}
\usepackage{rotating}
\usepackage{latexsym}
\usepackage{amsmath}
\usepackage{amssymb}
\usepackage{relsize}
\usepackage{geometry}
\geometry{letterpaper}
\usepackage{color}
\usepackage{bm}
\usepackage{slashed}
%\usepackage{showlabels}




%%%%%%%%%%%%%%%%%%%%%%%%%%%%%%%%%%%%%%%%%%%%%%%%%%%%%%%%%%%%%%%%%%%%%%%%%%%%%%%%
%                                                                              %
%                                                                              %
%                     D O C U M E N T   S E T T I N G S                        %
%                                                                              %
%                                                                              %
%%%%%%%%%%%%%%%%%%%%%%%%%%%%%%%%%%%%%%%%%%%%%%%%%%%%%%%%%%%%%%%%%%%%%%%%%%%%%%%%
\def\baselinestretch{1.1}
\renewcommand{\theequation}{\thesection.\arabic{equation}}

\hyphenation{con-fi-ning}
\hyphenation{Cou-lomb}
\hyphenation{Yan-ki-e-lo-wicz}




%%%%%%%%%%%%%%%%%%%%%%%%%%%%%%%%%%%%%%%%%%%%%%%%%%%%%%%%%%%%%%%%%%%%%%%%%%%%%%%%
%                                                                              %
%                                                                              %
%                      C O M M O N   D E F I N I T I O N S                     %
%                                                                              %
%                                                                              %
%%%%%%%%%%%%%%%%%%%%%%%%%%%%%%%%%%%%%%%%%%%%%%%%%%%%%%%%%%%%%%%%%%%%%%%%%%%%%%%%

\begin{document}

	Before you read my remarks inserted between your formulas below, please consider the following.
	
	[
	You don't have to perform the manipulations that I describe immediately, 
	just give it a read, it may clarify.
	I also should ask you to use antisymmetrization on indices in $ \partial_{[\mu} A_{\nu]} $
	and $ \partial_{[\mu} B_{\nu]} $ everywhere and in all occurences, or the coefficients will be wrong 
	]

\begin{itemize}
\item
	There are two auxiliary fields --- $ F_{\mu\nu} $ and $ G_{\mu\nu} $. 
	We want to eliminate them 
	(although we don't have to, as they are useful for Hamiltonian approach, but let us do)

\item
	In my January note I chose to eliminate them both at once. 
	There is no miscalculation in that note, and the result is 
	a self-dual QED with two gauge fields constrained by duality

\item
	We can choose to eliminate $ F_{\mu\nu} $ and $ G_{\mu\nu} $ one--by--one instead. 
	Fine. Choose any of them, they are equivalent, say $ G_{\mu\nu} $. 
	If you eliminate it, you will find that the other field --- $ F_{\mu\nu} $ ---
	enters only linearly. 
	That is  --- it is a Lagrange multiplier of the duality constraint.
	And the rest of the action will again be a QED, therefore, forced to be self-dual

\item
	Finally, as you have noticed, it may be convenient to introduce a {\it sum}
\[
	H_{\mu\nu}  ~~=~~  F_{\mu\nu}  ~~+~~  \widetilde{G}_{\mu\nu}\,.
\]
	I suggest to introduce a {\it difference} as well
\[
	M_{\mu\nu}  ~~=~~  F_{\mu\nu}  ~~-~~  \widetilde{G}_{\mu\nu}\,.
\]
	Now without any elimination you'll see that the difference $ M_{\mu\nu} $ acts
	as a Lagrange multiplier for the same duality constraint.
	Consequent elimination of $ H_{\mu\nu} $ again leads to the same result as above

\item
	The approach of your email (copied on the next page), is a kind of mix,
	where you introduce the sum $ H_{\mu\nu} $, while keeping $ G_{\mu\nu} $,
	instead of trading it for the difference. Fine.
	I argue below that it still certainly leads to the same result

\end{itemize}


	Dear  Dr. Bolokhov

	Sorry,  I  have misanderstood  a  statement  of  your  previous  mail; 
	I  thought that  your  claim  was  that  the  two  big  round  brackets  in eq. (6)  vanish  
	but  now  I see  that  you intended  that  say  cancel  the  last  (non  big)  brackets
	which  is  correct  if  one  use the field  eqs. (5).

	In any  case  your  approach  remains inconsistent.
	Indeed  your  lagrangian  (3)  can  be  rewritten  as 
\[
	L =  1/4 \partial_{\mu} A_{\nu}H^{\mu\nu} - (1/16)( H_{\mu\nu})^2
		- 1/4 ( \partial_{\mu}A_{\nu} - 1/2 \epsilon_{\mu\nu\rho\sigma}\partial^{\rho}B^{\sigma}) \tilde G^{\mu\nu}  
		+ k_{\mu}A^{\mu} + j_{\mu}B^{\nu}
\]
\textsl{\textbf{\footnotesize [you seem to be implying antisymmetrization on 
{\boldmath $ \partial_\mu A_\nu $} and {\boldmath $ \partial_\rho B_\sigma $} here.
	Otherwise the coefficients would be incorrect --- e.g. in the first term]}}\\
	where  $ H =  F + \tilde G $
	or,   integrating  over   $ H $,  
\[
	L =  (1/16 )(\partial_{[ \mu} A_{\nu ]})^2
	- (1/4 )( \partial_{\mu}A_{\nu} - 1/2 \epsilon_{\mu\nu\rho\sigma}\partial^{\rho}B^{\sigma}) \tilde G^{\mu\nu} 
	+ k_{\mu}A^{\mu} + j_{\mu}B^{\nu} 
\]
\textsl{\textbf{\footnotesize [Now, (implied) antisymmetrization is still missing in the second term,
while the first term has it, but its coefficient should be ``{\boldmath $ 1/4 $}'' as well]}}\\
	and  since 
\[
	\int ( 1/2 \epsilon_{\mu\nu\rho\sigma}\partial^{\rho}B^{\sigma}) \tilde A^{\mu\nu})  =  0  
\]
\textsl{\textbf{\footnotesize [Now this is crucial! While there is something wrong with notations in this formula, 
	I certainly do understand what you mean:
{\boldmath
\[
	\int 1/2\, \epsilon_{\mu\nu\rho\sigma}\, \partial^{\rho}B^{\sigma} \partial^\mu A^\nu ~~=~~ 0
\]}
	and this is not right.
	The integrand is a total derivative, which cannot be discarded in the presence of monopoles
]}}\\
	your  action  is  identical  to  the  action  
\[
	\int L =  \int[  - (1/4 )( \partial_{\mu}A_{\nu} - 1/2 \epsilon_{\mu\nu\rho\sigma}\partial^{\rho}B^{\sigma}) K^{\mu\nu}  + 
	k_{\mu}A^{\mu} + j_{\mu}B^{\nu}]
\]
\textsl{\textbf{\footnotesize [the first term vanishes, I agree, 
	but once the above ``topological'' integral is added back, everything comes into place]}}\\
	where  
\[
	K_{\mu\nu} =  \tilde G_{\mu\nu} - 1/4 (\partial_{[ \mu} A_{\nu ]})\,,
	\qquad\qquad\textsl{\textbf{\footnotesize [in fact there is no ``{\boldmath $1/4$}'']}}\\
\]
which  is  clearly  inconsistent.

With  my  best  regards

                                     Mario  Tonin

\vspace{1.0cm}
\centerline{* \qquad\qquad\qquad\qquad * \qquad\qquad\qquad\qquad *}
\vspace{1.0cm}


	You could ask how is this better to have $ F_{\mu\nu} $ and $ G_{\mu\nu} $
	if this or another way we are just enforcing the duality constraint by a Lagrange multiplier,
	the thing I wanted to avoid from the start?
	
	True, there is a Lagrange multiplier. 
	But I stress again that it carries physical sense --- first, being the fieldstrength tension
	--- and second, being the conjugate momentum, which is useful for applying the Hamiltonian approach.
	But let's leave this discussion for later because we just can't break through
	the difficulties with the initial formulas


\end{document}
